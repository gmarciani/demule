\section{Introduction}
\label{sec:introduction}


% %
% WORK INTRODUCTION
% %
As computing is getting more ubiquitous in our lives, computer infrastructures are getting increasingly complex and software applications are required to meet high level performance.

%
% %
% PROBLEM STATEMENT
% %
In this context, knowing how to design and optimize computer systems and networks is one of the most important skills for software engineers and a strategic asset for companies, both in terms of technology and investments.
%
% %
% APPROACH
% %
In this technical report we propose a next-event simulator to analyze the performance of a two-layers Fog-like system that serves classed workloads and leverages an off-loading policy between its layers.


% %
% PAPER ORGANIZATION
% %
The remainder of the paper is organized as follows.
%
In Section~\ref{sec:system} we give an high level description of the target system.
%
In Section~\ref{sec:random-number-generation} we describe the pseudo-random number generator adopted to generate random variates for the next-event simulation model.
%
In Section~\ref{sec:performance-modeling} we describe the the next-event simulation model in terms of goals, conceptual model, specification model, computational model, verification and validation.
%
In Section~\ref{sec:evaluation} we show the experimental results about both the randomness of the adopted pseudo-random number generator and the performance analysis of the target system conducted leveraging our simulator.
%
In Section~\ref{sec:usage} we show how to configure and run experiments and give some sample outputs to provide a better idea of what has been created.
%
In Section~\ref{sec:conclusions} we conclude the paper summing up the work that has been done and delineating future improvements.
