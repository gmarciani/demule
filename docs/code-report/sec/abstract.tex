\begin{abstract}
It has been previously proposed that understanding the mechanisms of
contour perception can provide a theory for why some flow-rendering
methods allow for better judgments of advection pathways than others.
In the present article, we develop this theory through a numerical
model of the primary visual cortex of the brain (Visual Area 1) where
contour enhancement is understood to occur according to most
neurological theories. We apply a two-stage model of contour
perception to various visual representations of flow fields evaluated
using the advection task of Laidlaw et al. [2001]. In the first
stage, contour {enhancement} is modeled based on Li's cortical model
[Li 1998]. In the second stage, a model of streamline {tracing} is
proposed, designed to support the advection task. We examine the
predictive power of the model by comparing its performance to that of
human subjects on the advection task with four different
visualizations. The results show the same overall pattern for humans
and the model. In both cases, the best performance was obtained with
an aligned streamline-based method, which tied with a LIC-based
method. Using a regular or jittered grid of arrows produced worse
results. The model yields insights into the relative strengths of
different flow visualization methods for the task of visualizing
advection pathways.
\end{abstract}
